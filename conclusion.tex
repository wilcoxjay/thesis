\chapter{Conclusion}
\label{chap:conclusion}

\Cref{chap:verdi,chap:disel,chap:mypyvy} presented programming languages techniques
  to address complexity in verifying distributed systems implementations.
Verdi and \disel both address forms of system complexity:
  Verdi separates fault tolerance mechanisms from application logic,
  while \disel separates reasoning between cooperating services.
While successful at reducing system complexity,
  both Verdi and \disel still suffer from high \emph{proof} complexity,
  in large part because of the need to manually state and prove inductive invariants.
This motivated us to build tools for automated verification and invariant inference
  on top of the \mypyvy toolbox.
In particular, we have used \mypyvy to implement the \updr algorithm,
  which can automatically infer certain classes of inductive invariants
  for symbolic transition systems.
When applicable, such automated tools greatly reduce proof effort and complexity.

Despite all this progress, it remains challenging to use verification tools
  to verify real protocols and implementations.
For example, if a system's inductive invariant cannot be expressed
  in the logic supported by the solver underlying \mypyvy,
  then \mypyvy will be of little use when verifying the system.
In this case, the simplification and automation that would come from using \mypyvy is not available.
This example shows that \mypyvy is
  a tool for \emph{simplifying away} a certain kind of complexity
  that comes from stating and proving inductive invariants.
In contrast, another approach is \emph{embracing complexity},
  to use a phrase coined by Jean Yang.\footnote{See \url{https://twitter.com/jeanqasaur/status/1389645922183696384}.}
Such tools make the verification engineer's life easier not by completely solving (and hiding) the problem,
  but rather by efficiently presenting the engineer with the information they need.
We look forward building complexity-embracing tools that go beyond what current automated solvers can handle.
