\chapter{\texttt{mypyvy}}


\section{Introduction}

\begin{verbatim}
- something something invariant inference
- something something clean research platform
\end{verbatim}

\section{Background on Transition Systems}

\begin{verbatim}
- the crawler
  - thanks to jon howell for suggesting this example
  - cute little robot starting at (0, 5) in the plane
  - there is a circular hole of radius 3 centered at the origin
  - crawler can take 1 step due north or one diagonal step south-east
  - prove: the crawler never falls in the hole
  - model steps as binary relation on "states" (points in the plane)
  - a state s is reachable if there is some sequence of steps that starts from
    (0, 5) and ends at s
  - <draw nice pictures>
  - inductive invariant vs reachable states vs bad states
- mathematical transition systems
  - incrementally formalize the crawler in set theory
  - a transition system is a state space, a set of initial states,
    and a binary transition relation on states
  - an execution of a transition system is a nonempty sequence of states,
    the first of which is initial, and each adjacent pair of which are
    related by the transition relation.
  - a reachable state is one that appears in some execution
  - alternatively,
  - an invariant of a transition system is a set of states that contains
    all reachable states
  - the way to prove something is an invariant is by induction on executions.
  - an inductive invariant I is a set of states that:
    - contains all initial states
    - is closed under the transition relation;
      ie, if s \in I and s -> s' then s' \in I
  - the "safety verification problem for transition systems" is the problem of,
    given a transition system and a "goal" invariant called the safety property,
    find an inductive invariant that implies the safety property.
- first-order symbolic transition systems "on paper"
  - incrementally formalize the crawler in logic
  - so far everything is in math/set theory
  - can formalize transition systems in first-order logic as follows
  - states will be first-order structures over some vocabulary \sigma
  - initial states will be described by a FO-formula over \sigma
  - transition relation is a FO-formula over the "double vocabulary"
    consisting of two copies of \sigma, the second copy we call "primed"
  - the safety property is also a (single vocabulary) formula
  - the goal is to find a (single vocabulary) formula that is an inductive
    invariant for the system
  - inductiveness can be checked by checking validity of:
    - Init => I       (note: single vocab)
    - I /\ TR => I'   (note: double vocab)
      where I' is I with all the vocabulary symbols replaced by the primed copy
\end{verbatim}

\section{The Crawler in \mypyvy}

\begin{verbatim}
- incrementally formalize the crawler in \mypyvy
- make a joke about "mutable constant"
\end{verbatim}

\begin{verbatim}
mutable constant x: int
mutable constant y: int

init x = 0 & y = 5

transition north()
  modifies y
  new(y) = y + 1

transition south_east()
  modifies x, y
  & new(x) = x + 1
  & new(y) = y - 1

safety x * x + y * y > 9

invariant x + y >= 5
\end{verbatim}

\section{Expressing Transition Systems in \mypyvy}

\begin{verbatim}
- sort
- immutable/mutable relation/constant/function
- init
- transition
- modifies clauses
- all the expressions, k-stateness
- zero/one/twostate definitions
- attributes
- typechecking/inference
- implicit quantification of capitalized vars at outer scope
- note on implicit existential on transition params, but *not* defn params
- note on modifies clauses/frame conjuncts
\end{verbatim}

\section{Queries on Transition Systems}

\begin{verbatim}
- trace/bmc; the trace declaration; sat/unsat qualifier
- how to read states printed by mypyvy
- verify: invariant/safety
- zero/one/twostate theorem
- side note: custom printers using attributes
- updr
\end{verbatim}

\section{Internals of \mypyvy}

\begin{verbatim}
- a tour of main()
- the mental model of k-state formulas (correctly handling immutable)
  - evaluating a k-state formula on a trace
- philosophy on interacting with z3, the Solver class
- how to write a mypyvy "plugin"
- syntax.the_program and its consequences
\end{verbatim}

\section{Past, Present, and Future Work}

\begin{verbatim}
- our port of the raft proof
- yotam's cav19
- jason's plid20
- pd
- derived relations?
- yotam's looking back algorithm or whatever it's called
- future directions in internals:
  - collapsing the many kinds of queries into one
  - handling transition declarations more uniformly (exists, modifies)
  - introducing a "logic" layer or other IR, or do less at z3 level
  - revisiting the "one program" mindset
\end{verbatim}

\section{Conclusion}

\begin{verbatim}
- call to arms for collaborators, builders-on-toppers, and users
- vision blah blah about verification UX and "exploration" of a TS,
  saving progress from run to run, "workbench"
\end{verbatim}