\chapter{\mypyvy}

% Put this somewhere
%  % The project started because I wanted a clean sandbox in which
%  % to implement the \updr invariant inference algorithm~\cite{updr-jacm}.

\section{Introduction}

\begin{verbatim}
- something something invariant inference
- something something clean research platform
\end{verbatim}

\section{Background on Transition Systems}

\begin{verbatim}
- the crawler
  - thanks to jon howell for suggesting this example
  - cute little robot starting at (0, 5) in the plane
  - there is a circular hole of radius 3 centered at the origin
  - crawler can take 1 step due north or one diagonal step south-east
  - prove: the crawler never falls in the hole
  - model steps as binary relation on "states" (points in the plane)
  - a state s is reachable if there is some sequence of steps that starts from
    (0, 5) and ends at s
  - <draw nice pictures>
  - inductive invariant vs reachable states vs bad states
- mathematical transition systems
  - incrementally formalize the crawler in set theory
  - a transition system is a state space, a set of initial states,
    and a binary transition relation on states
  - an execution of a transition system is a nonempty sequence of states,
    the first of which is initial, and each adjacent pair of which are
    related by the transition relation.
  - a reachable state is one that appears in some execution
  - alternatively,
  - an invariant of a transition system is a set of states that contains
    all reachable states
  - the way to prove something is an invariant is by induction on executions.
  - an inductive invariant I is a set of states that:
    - contains all initial states
    - is closed under the transition relation;
      ie, if s \in I and s -> s' then s' \in I
  - the "safety verification problem for transition systems" is the problem of,
    given a transition system and a "goal" invariant called the safety property,
    find an inductive invariant that implies the safety property.
- first-order symbolic transition systems "on paper"
  - incrementally formalize the crawler in logic
  - so far everything is in math/set theory
  - can formalize transition systems in first-order logic as follows
  - states will be first-order structures over some vocabulary \sigma
  - initial states will be described by a FO-formula over \sigma
  - transition relation is a FO-formula over the "double vocabulary"
    consisting of two copies of \sigma, the second copy we call "primed"
  - the safety property is also a (single vocabulary) formula
  - the goal is to find a (single vocabulary) formula that is an inductive
    invariant for the system
  - inductiveness can be checked by checking validity of:
    - Init => I       (note: single vocab)
    - I /\ TR => I'   (note: double vocab)
      where I' is I with all the vocabulary symbols replaced by the primed copy
\end{verbatim}

\section{The Crawler in \mypyvy}

\begin{verbatim}
- incrementally formalize the crawler in \mypyvy
- make a joke about "mutable constant"
\end{verbatim}

\begin{verbatim}
mutable constant x: int
mutable constant y: int

init x = 0 & y = 5

transition north()
  modifies y
  new(y) = y + 1

transition south_east()
  modifies x, y
  & new(x) = x + 1
  & new(y) = y - 1

safety x * x + y * y > 9

invariant x + y >= 5
\end{verbatim}

\section{Expressing Transition Systems in \mypyvy}

\begin{verbatim}
- sort
- immutable/mutable relation/constant/function
- init
- transition
- modifies clauses
- all the expressions, k-stateness
- zero/one/twostate definitions
- attributes
- typechecking/inference
- implicit quantification of capitalized vars at outer scope
- note on implicit existential on transition params, but *not* defn params
- note on modifies clauses/frame conjuncts
\end{verbatim}

\section{Queries on Transition Systems}

\begin{verbatim}
- trace/bmc; the trace declaration; sat/unsat qualifier
- how to read states printed by mypyvy
- verify: invariant/safety
- zero/one/twostate theorem
- side note: custom printers using attributes
- updr
\end{verbatim}

\section{Internals of \mypyvy}

\begin{verbatim}
- a tour of main()
- the mental model of k-state formulas (correctly handling immutable)
  - evaluating a k-state formula on a trace
- philosophy on interacting with z3, the Solver class
- how to write a mypyvy "plugin"
- syntax.the_program and its consequences
\end{verbatim}

\section{Using \mypyvy}

\begin{verbatim}
- our port of the raft proof
- yotam's cav19
- jason's plid20
- pd
- derived relations?
- yotam's looking back algorithm or whatever it's called
\end{verbatim}

\section{Related Work}

\mypyvy is directly inspired by \ivy~\cite{Padon-al:PLDI16}.\footnote{
  \ivy's code is available on Github at \url{https://github.com/kenmcmil/ivy}.
  % 
  See also the \ivy website at \url{http://microsoft.github.io/ivy/}.
}
%
The \ivy tool supports specification, implementation, and verification of systems,
including distributed and concurrent systems.
%
Systems are expressed as a set of \emph{actions},
each written in a simple imperative programming language
over state variables from a first-order vocabulary.
%
The verification queries \ivy asks of the underlying solver
are carefully designed to land in a decidable fragment of logic,
increasing the efficiency, reliability, and predictability
of the verification.
%
When verification fails, concrete counterexample traces
are shown to the user demonstrating the violation.
%
\ivy{} also has a powerful module system
that supports so-called ``circular'' assume-guarantee reasoning,
where all modules get to assume all other modules' invariants,
but are under the obligation to show that they do not violate
their own invariants.
%
(This reasoning is not actually circular, but sound,
because ``nobody violates their invariants first''
implies ``nobody violates their invariants''.)

One can view \mypyvy as similar to a hypothetical intermediate language
in \ivy{}'s pipeline to the solver.
%
\ivy{} compiles the modular imperative program
into a set of purely logical transition systems,
each of which must be verified.
%
One could imagine making this connection explicit,
by using \mypyvy as a ``backend'' for \ivy,
translating the transition systems into \mypyvy syntax.
%
This would have the advantage of making
\mypyvy's invariant inference algorithms available
to \ivy programs.
%
Indeed, many of the examples and benchmarks used by \mypyvy
were manually translated from \ivy.
%
We have begun work on such a translator,
and hope to continue to work more closely with \ivy in the future.


Dafny is a programming language
designed from the ground up for verification~\cite{Leino:LPAR10}.
%
Dafny is built on top of the Boogie intermediate verification language~\cite{boogie-manual},
which itself uses the Z3 SMT solver~\cite{z3}.
%
Dafny has an imperative object-oriented programming language with
objects, statements, loops, arrays, and heap-allocated data structures,
and has a rich Hoare logic for reasoning about these programs.
%
It also has a purely functional expression-oriented programming language
with first-class and higher-order functions, recursion, lists and logical quantifiers.
%
These logical features can be used express the specification of the imperative code,
or they can be used by themselves, turning Dafny into more of a proof assistant
than a programming language.
%
Dafny enjoys a high degree of proof automation, since all obligations are
eventually sent to Z3.
%
However, these queries can have complex quantifier structure to them, meaning
that they typically do not fall into any decidable fragment of first-order logic.
%
Dafny instructs Z3 to use syntactic heuristics based on E-matching
to manage quantifier instantiation process~\cite{z3-e-matching,simplify}.
%
This approach achieves good performance in practice,
but it means that the solver cannot return counterexamples,
and that the user must have a basic understanding of E-matching
in order to be an expert user of quantifiers in Dafny.


\begin{verbatim}
- CMP method / Cadence SMV
  - https://link.springer.com/chapter/10.1007%2F978-3-540-30494-4_27
  - https://www.markrtuttle.com/data/papers/talupur-tuttle-fmcad08.pdf
  - https://link.springer.com/article/10.1007%2Fs10703-010-0092-y
  - https://www.semanticscholar.org/paper/Parametrized-System-Verification-with-Guard-and-Krstic/70e7b7d15fe569d05f021884eef32eea3e07e82b
- VMT http://www.vmt-lib.org/
- nuXmv
- AVR: https://github.com/aman-goel/avr
- TLA+
- Coq
- avy
- pono
- abc

- something about k-state formulas and semantics over finite traces (Vardi's paper)
\end{verbatim}

\section{Future Work}

\begin{verbatim}
- future directions in internals:
  - collapsing the many kinds of queries into one
  - handling transition declarations more uniformly (exists, modifies)
  - introducing a "logic" layer or other IR, or do less at z3 level
  - revisiting the "one program" mindset
\end{verbatim}

\section{Conclusion}

\begin{verbatim}
- call to arms for collaborators, builders-on-toppers, and users
- vision blah blah about verification UX and "exploration" of a TS,
  saving progress from run to run, "workbench"
\end{verbatim}