\chapter{\texttt{mypyvy}}


\section{Introduction}

\begin{verbatim}
- mathematical transition systems
- first-order symbolic transition systems "on paper"
\end{verbatim}

\section{Expressing Transition Systems in \mypyvy}

\begin{verbatim}
- sort
- immutable/mutable relation/constant/function
- init
- transition
- zero/one/twostate definitions
- attributes
- typechecking/inference
- implicit quantification of capitalized vars at outer scope
- note on implicit existential on transition params, but *not* defn params
- note on modifies clauses/frame conjuncts
\end{verbatim}

\section{Queries on Transition Systems}

\begin{verbatim}
- trace/bmc; the trace declaration; sat/unsat qualifier
- how to read states printed by mypyvy
- verify: invariant/safety
- zero/one/twostate theorem
- side note: custom printers using attributes
- updr
\end{verbatim}

\section{Internals of \mypyvy}

\begin{verbatim}
- a tour of main()
- the mental model of k-state formulas (correctly handling immutable)
  - evaluating a k-state formula on a trace
- philosophy on interacting with z3, the Solver class
- how to write a mypyvy "plugin"
- syntax.the_program and its consequences
\end{verbatim}

\section{Past, Present, and Future Work}

\begin{verbatim}
- our port of the raft proof
- yotam's cav19
- jason's plid20
- pd
- derived relations?
- yotam's looking back algorithm or whatever it's called
- future directions in internals:
  - collapsing the many kinds of queries into one
  - handling transition declarations more uniformly (exists, modifies)
  - introducing a "logic" layer or other IR, or do less at z3 level
  - revisiting the "one program" mindset
\end{verbatim}

\section{Conclusion}

\begin{verbatim}
- call to arms for collaborators, builders-on-toppers, and users
- vision blah blah about verification UX and "exploration" of a TS,
  saving progress from run to run, "workbench"
\end{verbatim}